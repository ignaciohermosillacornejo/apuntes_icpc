\documentclass[../Notes/main.tex]{subfiles}

\begin{document}
\section{Mathematics}
\subsection{Useful Data}
\begin{tabular}{| l | c | c | c |}
    \hline
    \(n\)     & Primes less than \(n\) & Maximal Prime Gap & \(\max_{0<i<n}(d(i))\) \\
    \hline
    \num{1e2} & \num{25}               & \num{8}           & \num{12}               \\
    \num{1e3} & \num{168}              & \num{20}          & \num{32}               \\
    \num{1e4} & \num{1229}             & \num{36}          & \num{64}               \\
    \num{1e5} & \num{9592}             & \num{72}          & \num{128}              \\
    \num{1e6} & \num{78498}            & \num{114}         & \num{240}              \\
    \num{1e7} & \num{664579}           & \num{154}         & \num{448}              \\
    \num{1e8} & \num{5761455}          & \num{220}         & \num{768}              \\
    \num{1e9} & \num{50487534}         & \num{282}         & \num{1344}             \\
    \hline
\end{tabular}


\subsection{Modular Arithmetic}

\subsubsection{Chinese Remainder Theorem}
\lstinputlisting[firstline=2]{CRT/CRT.cpp}

\subsubsection{Binomial Coefficients mod m}
\lstinputlisting[firstline=2]{nCr/nCr.cpp}

\subsection{Primality Checks}
\subsubsection{Miller Rabin}
\lstinputlisting[firstline=2]{primalityChecks/millerRabin/millerRabin.cpp}

\subsubsection{Sieve of Eratosthenes}
\lstinputlisting[firstline=2]{primalityChecks/sieveEratosthenes/sieve.cpp}

\subsubsection{trialDivision}
\lstinputlisting[firstline=2]{primalityChecks/trialDivision/trialDivision.cpp}

\subsection{Others}
\subsubsection{Polynomials}
\lstinputlisting[firstline=2]{polynomials/polynomials.cpp}

\subsubsection{Factorial Factorization}
\lstinputlisting[firstline=2]{factorialFactorization/factorialFactorization.cpp}

\end{document}